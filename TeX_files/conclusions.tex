\documentclass[../main.tex]{subfiles}

\begin{document}

\chapter{Conclusions}

In conclusion, we have shown that it is possible to create a highly modular and extensible Natural Language Interface to the Semantic Web that supports the use of prepositional phrases in queries using our approach.

We presented a novel event-based denotational semantics, UEV-FLMS, that improves on EV-FLMS, unifying
the treatment of several semantic concepts, solving two problems with the original semantics, and expanding the capabilities of prepositional phrases.
In addition to this, we demonstrated a novel way of handling the word ``by'', as in ``discovered by'' in our semantics by treating it directly as a preposition.
We integrated this semantics with a parser constructed as an executable attribute grammar, extending the work by Frost and Hafiz in 2008\cite{frosthafiz2008} to support
monadic values.  We showed that our approach was viable by performing two benchmarks in Chapter \ref{chapter:timing}, and improving on the asymptotic time-complexity of the original semantics with our \texttt{condense} function.  We discussed potential methods to improve efficiency further in Chapter \ref{chapter:futurework}.  We also uploaded our work to Hackage, an online repository of Haskell packages, in the form of the XSaiga package.  Finally, we built an online query interface to this program, creating a highly modular and extensible Natural Language Interface to the Semantic Web that supports complex chained prepositional phrases in queries.

The approach used in this thesis for handling the word ``by'' could potentially apply to other related problems in Natural Language Processing.
In unifying the treatment of distinct semantic concepts, it may be possible to find simpler ways of handling linguistic concepts which have been inherently difficult
to capture in formal semantics.  It also could be used to cleanly handle the separation of Primary and Secondary sources, and future semantics may be able to infer
the truth of statements by agreement among Secondary sources.  For example, in the statement ``Sally said that John thinks the moon is made of cheese'',
In the absence of the definitive statement to the contrary from a Primary source, e.g. ``John denied thinking that the moon is made of cheese'', it may be reasonable
to infer that John believes that the moon is made of cheese, especially if Sally is a particularly trustworthy source.  Research in this area will
become increasingly important in order to assess the trustworthiness of results from queries in the Semantic Web.  It may also be possible to apply our technique
for handling prepositional phrases to handling language constructs seen in other languages, such as postpositions and circumpositions.

One area where our research could be particularly relevant is in constructing Natural Language Interfaces for IoT-enabled devices.  It could be feasible to provide an interface to control a variety of these devices using our apporach, improving accessibility for users who suffer disabilities.  As the Semantic Web becomes more mainstream, there will be an increasing need for enabling technologies like these.  It is our hope that researchers in the future will consider building on our approach in order to fulfill this growing need.

\end{document}