\documentclass[../main.tex]{subfiles}

\begin{document}

\chapter{The Query Program}

\label{chapter:implementation}

\section{Implementation language}

We chose to implement UEV-FLMS in the Haskell programming language\cite{haskell}.
Briefly, Haskell is a lazily-evaluated functional programming language with first class support
for monads.  We summarize these concepts and provide justification for our choice in the following subsections:

\subsection{Functional Programming}

Functional programming languages are declarative, rather than imperative, in nature.
This means that programs are written by composing functions together, much like how mathematical
functions can be defined in terms of function composition.  Unlike imperative languages,
mutable state is avoided in programs written in functional languages. This lends lends itself to a highly expressive
style of programming that heavily encourages code reuse while still retaining the ability to easily reason about code.


\subsection{Lazy-evaluation}

In a language that is {\em lazily evaluated}, expressions and subexpressions are evaluated only as they are needed at run-time.  This enables infinitely recursive data structures to be expressed directly in the language, and allows for improved performance in some cases by leaving unnecessary calculations unevaluated.

For example, consider the function \texttt{primes} that returns a list of all prime numbers.
If no elements of \texttt{primes} are used, then no computation is performed at all.
If only the first 20 elements of \texttt{primes} are used, only the first 20 elements need to be evaluated,
with the rest being left unevaluated.  In an eagerly evaluated language, every element of the list returned by \texttt{primes} would be computed the first time it was referenced, resulting in a non-terminating loop.
In this example, lazy-evaluation enables users to use an intuitive abstraction to iterate through a sequence of prime numbers while avoiding the costs associated with those abstractions in eagerly evaluated languages.

\subsection{Monads}

Monads in functional programming are inspired from their counterparts in category theory.
In a functional programming language, a type $A$ is a monad if there exists definitions of two functions in that language with the types of those functions as follows:

\begin{itemize}
	\item $\mathtt{bind} : A\ x \rightarrow \left(\lambda x \rightarrow A\ y\right) \rightarrow A\ y$
	\item $\mathtt{return} : x \rightarrow A\ x$
\end{itemize}

\newcommand{\hbind}{>\!\!>\!\!=}

The \texttt{bind} function is commonly used as a binary operator.  In Haskell, the bind function is denoted with operator $\hbind$.
In addition, these definitions must satisfy the {\em monad laws}.  We use this syntax to summarize the monad laws as follows.
Here, $a \equiv b$ denotes that the expression $a$ is semantically equivalent to the expression $b$ and hence they are interchangeable.
These laws are defined using lambda calculus.

\subsubsection{Monad Laws}

$(\forall x)(\forall y)(\forall f)(\forall m)(\forall a)$ $f$ is a function with type $f : x \rightarrow A\ y$, $m$ is a value with type $m : A\ x$, and $a$ is a value with type $a : x$

\begin{itemize}
	\item Left-identity: $\mathtt{return}\enspace a\ \hbind\ f\ \equiv\ f\enspace a$
	\item Right-identity: $m\ \hbind\ \mathtt{return}\ \equiv\ m$
	\item Associativity: $(m\ \hbind\ f)\ \hbind\ g\ \equiv\ m\ \hbind\ (\lambda x \rightarrow f\enspace x\ \hbind\ g)$
\end{itemize}

In Haskell, computations with monads are expressible with a special syntax called the ``do'' syntax.
This allows for a convenient way of expressing monadic computations directly within the language.
Haskell gives special treatment to the monad \texttt{IO}, enabling computations within it
to be non-referentially transparent.  An example expression in Haskell using the IO monad is as follows:

\begin{code}
main :: IO ()
main = do
	line <- readLine
	putStrLn ("Hello world! " ++ line)
\end{code}

This syntax is equivalent to the code:

\begin{code}
main :: IO ()
main = readLine >>= (\line -> putStrLn ("Hello world! " ++ line))
\end{code}

Monads are useful to express computations that involve multiple independent actions with an implicit 
action that ``ties'' them together.  This can be used, for example, to implicitly pass state as an argument through
functions, a technique used in Frost and Hafiz's parser to efficiently parse highly ambiguous left recursive grammars\cite{frosthafiz2008}.

\subsection{Why Haskell?}

In particular, mathematical functions are easily implemented in functional languages.  Due to their declarative nature,
often times the definition of the mathematical function itself can be directly expressed in the language.  Since UEV-FLMS
is heavily rooted in set-relational theory\cite{frost1989constructing}, we were able to implement all of our semantic functions as they were defined in Chapter \ref{chapter:semantics} by directly expressing those functions in Haskell.  To accommodate communication with external triplestores, we ``lifted'' the implementations of the semantic functions into the IO monad, enabling our semantics to use non-referentially transparent \texttt{getts} functions.  We were also able to modify the original parser that Frost and Hafiz described\cite{frosthafiz2008} to support non-referentially transparent attributes and integrate that with our semantics, owing to the reusability of code that functional languages encourage.

It certainly would have been possible to use an imperative language to achieve the same end-result, however significantly more work would have been involved in doing so.  For one, implementation of UEV-FLMS semantics in an imperative language would have taken much more code, as the mathematical definitions themselves would not be directly expressible in the language. Another difficulty would have been in parsing English queries, since adapting an existing parser for this end may not have been possible.  Certainly, one would not be able to directly express an Executable Attribute Grammar directly in an imperative language without significant work, meaning that the semantics and the parser would not be able to be defined together in a piecewise fashion.

\section{Data representation}

Like the original Haskell implementations, we represent sets in our implementation using lists and binary relations as association lists.  We represent triples using 3-tuples and we represent URIs using the \texttt{String} type, which is a list of \texttt{Char}.

\section{Structure}

A graph representing the structure of the modules in the XSaiga package in relation to one another is presented in Figure \ref{fig:modulegraph}.  Briefly, there exists an arrow from module \texttt{A} to module \texttt{B} in the graph if and only if module \texttt{A} imports module \texttt{B} in the source code.

\begin{figure}[h]
	\centering
	\frame{\includegraphics[width=\columnwidth]{modules2}}
	\caption{Module graph of the XSaiga package}
	\label{fig:modulegraph}
\end{figure}


\section{AGParser2 and TypeAg2 modules}
\label{section:nonrefparserimpl}

The parser is defined in the \texttt{AGParser2} module.

\begin{figure}[h]
	\centering
	\frame{\includegraphics[width=\textwidth]{semanticsslide3}}
	{\scriptsize URL: \url{http://cs.uwindsor.ca/~richard/semantics_presentations/talk_for_GraphQ.ppt}}
	\caption{Parser operation: how an English sentence is mapped to semantic functions for evaluation\cite{frost2014demonstration}}
\end{figure}

The types of the semantic functions in \texttt{SolarmanTriplestore} are defined in the TypeAg2 module.
These are the implementations of the semantic functions described in Chapter \ref{chapter:semantics}.
The TypeAg2 module therefore serves as the interface between the parser and the semantics.

The changes made to the original parser to accommodate monadic code are summarized below:

\subsection{PrettyPrinting}
Utility functions for formatting the parser results had to be modified.

\subsubsection{Pure version}
In the pure version, we have:

\begin{code}
class PP' a where
pp' :: a -> Doc
\end{code}

where \texttt{Doc} is a formatted \texttt{String} and instances are defined for the various types
used by the parser itself.

\subsubsection{Monadic version}

In the monadic version, we have:

\begin{code}
class PP' a where
pp' :: a -> IO Doc
\end{code}

Here, the \texttt{pp'} function returns an IO action rather than a pure value.
All instances of \texttt{PP'} were modified accordingly.

\subsection{formatAttsFromAlt}

\subsubsection{Pure version}
\begin{code}
formatAttsFinalAlt  key e t  =
[(pp' [vcat [(vcat [vcat [vcat [text (show ty1v1)  |ty1v1<-val1]
|(((st,inAtt2),(end,synAtts)), ts)<-rs, end == e]                             
| ((i,inAt1),((cs,ct),rs)) <- sr ])
| (s,sr) <- t, s == key ] 

\end{code}

\subsubsection{Monadic version}
\begin{code}
formatAttsFinalAlt  key e t  =
sequence [(sequence [liftM vcat $ sequence
[(liftM vcat $ sequence
[liftM vcat $ sequence
[liftM vcat $ sequence
[liftM text (showio ty1v1) | ty1v1<-val1]
|(id1,val1)<-synAtts]] )
|(((st,inAtt2),(end,synAtts)), ts)<-rs, end == e]
| ((i,inAt1),((cs,ct),rs)) <- sr ])>>= pp'
| (s,sr) <- t, s == key ]
\end{code}

\section{Main module}

The \texttt{Main} module implements a CGI interface for evaluating Natural Language queries using the \texttt{SolarmanTriplestore} module.

\section{Interactive module}

The \texttt{Interactive} module is used by the Direct Query Interface to directly evaluate semantic functions.  It is intended to be used with SafeHaskell
in order to restrict the evaluation of functions to a trusted subset, suitable for online interfaces.

In \texttt{SolarmanTriplestore}, a dictionary is defined that maps words to semantic functions.  This module defines variables that are named
after those words such that those functions can be directly accessed in a Haskell interpreter.  This enables, for instance,

\begin{code}
hall $ discovered phobos
\end{code}

to be directly evaluated at a Haskell prompt.

A Haskell file InteractiveGenerator.hs is used to generate this module using the dictionary in \texttt{SolarmanTriplestore}.

\section{LocalData module}

This module contains an in-program version of the triplestore located on our SPARQL endpoint.  As the \texttt{Getts} module provides
a general interface to triplestores in the form of a typeclass, we are able to support both in-program triplestores as well as remote triplestores.
The module exports the list of triples as the variable \texttt{localData}:

\begin{code}
localData = [("event1000", "object", "sol"),
("event1000", "subject", "mercury"),
("event1000", "type", "orbit_ev"),
("event1001", "object", "sol"),
("event1001", "subject", "venus"),
("event1001", "type", "orbit_ev"),
("event1002", "object", "sol"),
("event1002", "subject", "earth"),
("event1002", "type", "orbit_ev"),
("event1003", "object", "sol"),
... ]
\end{code}

\section{SolarmanTriplestore and Getts modules}

The implementation of our semantics in Haskell is contained within these modules, along with the parser constructed as an executable attribute grammmar and
the dictionary used by the parser.  We detail our implementation improvements over EV-FLMS in Section \ref{section:improvementsimpl}.

\section{Improvements over Original Haskell Implementations}
\label{section:improvementsimpl}

\subsection{Type safety}

The original semantics were implemented as {\em pure functions} in Haskell, which was acceptable for the in-program triplestores they were used on.

In \cite{agboola2015extensible}, the \texttt{getts\_*} functions were modified to retrieve triples from external SPARQL endpoints, enabling the original
semantics to work with SPARQL triplestores.  For SPARQL triplestores, however, the \texttt{getts\_*} functions as defined in \cite{agboola2015extensible} are not actually guaranteed to be referentially transparent. In particular, SPARQL triplestores are able to change over time with triples being potentially added, removed, or modified.
For example, consider the query ``which people discovered a moon that was discovered by a person''.
``people'' and ``person'' are synonyms in our semantics and therefore the same query would be performed twice.  If the SPARQL triplestore changed in between these evaluations,
then these queries could return different results, violating referential transparency.

The function \texttt{unsafeDupablePerformIO} was used in \cite{agboola2015extensible} to force Haskell to treat the \texttt{getts\_*} functions as pure functions
in order to maintain compatibility with the original semantics. The problem with \texttt{unsafeDupablePerformIO} is that it subverts the type system of Haskell.  Code that is built using it is therefore not on
solid theoretical ground within the constructs of the language, and surprising effects can occur as a result.  The use of \texttt{unsafeDupablePerformIO}, while
legitimate in some cases, is heavily discouraged within the Haskell community\cite{tlmvconsensus}.

In this Thesis, we chose a different approach to handling external triplestore queries by representing the triplestore functions and semantics in terms of {\em monadic functions}.
By expressing the semantics and triplestore functions monadically, we stay safely within the confines of Haskell's type system, avoiding
the need to use \texttt{unsafeDupablePerformIO} in order to perform queries to external triplestores as a result.  Another key benefit of this approach is that it preserves the compositional nature of the original semantics.

The semantics are implemented in the IO monad currently.  However, if other monads were desired, just as with the parser described in Chapter \ref{chapter:combinators}, only minimal changes would be required in order to accommodate other instances of the Monad typeclass.  In Chapter \ref{chapter:futurework}, one potential application of this functionality is discussed.

\subsection{The Getts module: A generic interface to triplestores using a typeclass}

Typeclasses are used in Haskell to enable {\em ad-hoc polymorphism} in the definition of functions in the language.  This can be used
to provide generic interfaces to different types, without callers needing to be aware of the differences between those types.
We used this feature of the language to provide a generic interface for triplestores in the form of typeclass \texttt{TripleStore}.

\texttt{TripleStore m} subsumes the functionality that the \texttt{getts\_*} functions provided in the original semantics.

\begin{code}
class TripleStore m where
getts_1 :: m -> (Event, String, String) -> IO [String]
getts_2 :: m -> (Event, String, String) -> IO [String]
getts_3 :: m -> (Event, String, String) -> IO [String]

getts_fdbr_entevprop_type :: m -> String -> String -> IO FDBR
getts_fdbr_entevprop_type ev_data ev_type entity_type = do
evs <- getts_1 ev_data ("?", "type", ev_type)
getts_fdbr_entevprop ev_data entity_type evs

getts_fdbr_entevprop :: m -> String -> [Event] -> IO FDBR
getts_fdbr_entevprop ev_data entity_type evs = do
pairs <- liftM concat $ mapM (\ev -> do
ents <- getts_3 ev_data (ev, entity_type,"?")
return $ zip ents (repeat ev)) evs
return $ collect pairs

getts_members :: m -> String -> IO FDBR
getts_members ev_data set = do
evs_with_set_as_object <- getts_1 ev_data ("?", "object", set)
evs_with_type_membership <- getts_1 ev_data
("?", "type", "membership")
getts_fdbr_entevprop ev_data "subject" $
intersect evs_with_set_as_object evs_with_type_membership
\end{code}

First and foremost, the \texttt{getts\_*} functions defined in \texttt{TripleStore m} now properly return {\em IO actions}.
Three new functions are introduced: \texttt{getts\_fdbr\_entevprop},\\ \texttt{getts\_fdbr\_entevprop\_type}, and \texttt{getts\_members}.
\texttt{getts\_fdbr\_entevprop\_type} serves the same purpose that \texttt{make\_image} had in the original semantics.
These functions are named after their counterparts in Chapter 3.

Only the three \texttt{getts\_*} functions must be defined for the new type of triplestore at minimum. However efficient implementations
of all functions in the typeclass may be provided if desired.  We provide a backend using a SPARQL endpoint as a triplestore using the
\texttt{SPARQLBackend} type and a backend for in-program triplestores using the \texttt{[Triple]} type as instances of \texttt{TripleStore}.

\subsubsection{Basic query fusion}

In addition to this, a basic form of query fusion has been implemented in the form of memoization.  Briefly, queries and their results
are stored in {\em key-value stores}.  When a query is performed, it is first checked against the appropriate key-value store to see
if the same triplestore query has been made previously.  If it has, the previous result is returned.  Otherwise, the request is made to the remote
triplestore and its result is saved into the appropriate key-value store.  Multiple requests for the same information to the remote triplestore are therefore fused together.

The key-value stores are held in top-level mutable variables.  Defining top-level mutable variables in Haskell is a subject that has been explored in depth, with many proposals in how to provide an idiomatic method in the language to express it.  According to the Haskell community, the accepted way of doing this for now is with the following pattern\cite{tlmvconsensus}:

\begin{itemize}
	\item A top-level mutable variable \texttt{v} is defined using \texttt{v = unsafePerformIO \$ newIORef value} 
	\item \texttt{v} must be annotated with the compiler pragma \texttt{\{\--\# NOINLINE v \#\--\}}
\end{itemize}

\subsubsection{Efficiency of ``collect''}

\label{subsubsection:collect}
The \texttt{collect} function as defined in \cite{agboola2015extensible} used a key-value store from the \texttt{Data.Map.Lazy} module in Haskell to efficiently construct Images from relations represented as association lists.  Because all key-value pairs of the \texttt{Map} will be traversed in order, immediately and in all cases, in this thesis the \texttt{Data.Map.Strict} module was used instead.  The asymptotic time complexity of both methods are identical, however the \texttt{Strict} version uses less memory and is slightly faster as it does not attempt to store partially evaluated areas of the Map as {\em thunks}, which are partially evaluated Haskell expressions.

The \texttt{collect} function is defined in this module as:

\begin{code}
collect = Map.toList . Map.fromListWith (++) . map (\(x, y) -> (x, [y]))
\end{code}

%\begin{theorem}
%	\texttt{collect} is optimal.
%\end{theorem}

%Proof: Suppose a function \texttt{collect'} exists that converts a relation into an Image in faster than O($n$ lg $n$) time.

%Let $unsorted$ be an unsorted list whose list elements are an instance of the \texttt{Comparable} typeclass.
%That is, the list elements can be compared with one another using the comparison operators.

%Define the function \texttt{sort'} as follows:

%\begin{code}
%	sort' = uncollect . collect'
%	where
%	uncollect = map (\(ent, evs) -> map (\ev -> (ent, ev)) evs)
%\end{code}

\subsubsection{Efficiently constructing FDBRs from grouped association lists}
In addition, a new function, \texttt{condense}, has been created to efficiently construct $FDBR(r)$ such that $r$ is represented by a grouped association list.

\label{subsubsection:condense}
\begin{definition}[Grouped Association List]
	%Elements themselves are contiguous
	An association list where the indices of all pairs in the list with equal first components are contiguous
	An association list where all pairs with equal first components are contiguous in the list
\end{definition}

We refer to any association list that is not a grouped association list as an {\em ungrouped association list}.

The \texttt{condense} function is defined in this module as:

\begin{code}
condense =
map (\list -> (fst $ head list, map snd list)) . List.groupBy cmp
where
cmp x y = (fst x) == (fst y)
\end{code}

In our SPARQL backend, the \texttt{getts\_fdbr\_entevprop} and \texttt{getts\_fdbr\_entevprop\\\_type} functions efficiently construct FDBRs from their ENTEVPROP relations using this function by requesting to the SPARQL endpoint that the ENTEVPROP query results be sorted according to the first element in each pair.  Since this groups pairs together in the association list by their first element, \texttt{condense} is used instead of \texttt{collect} in order to construct the FDBR in the SPARQL backend.

The \texttt{condense} algorithm can be expressed in functional language pseudocode as follows.
In this pseudocode, we denote lists with the syntax $[a_1, a_2, \dots, a_n]$, with the empty list
denoted as $[]$.  Lists are represented as a recursive abstract datatype using the \texttt{cons} function as originally used in lambda calculus
to represent lists using Church Encoding\cite{jansen2013programming}.  The syntax $[a_1, a_2, \dots, a_n]$ is hence equivalent to the expression: \[ \mathtt{cons}\ a_1\ (\mathtt{cons}\ a_2\ (\dots (\mathtt{cons}\ a_n\ [])))) \]

We denote tuples with the syntax $(a_1, a_2, \dots, a_n)$.
Function application is denoted as in Chapter \ref{chapter:semantics}.
For example: \texttt{head [x, y, z] => x} and \texttt{tail [x, y, z] => [y, z]}.
Function definition is denoted with a similar syntax, supporting {\em pattern matching} on abstract datatypes such
as tuples and lists.  For example, the function \texttt{f (cons x xs) = x} matches its first argument with the outer \texttt{cons} in a list, and the function \texttt{g (x, y) = x} matches its first argument with a pair, assigning variables \texttt{x} and \texttt{y} to the first and second components of that pair, respectively.  A \texttt{where} clause may be appended to a function definition to define variables used within that function.  A let binding, using the form \texttt{let pattern = expr in expr'}, creates a local variable binding using pattern matching in an expression.  The expression \texttt{expr} is assigned to the pattern \texttt{pattern}, whose variables are made available in \texttt{expr'}.
Finally, conditional evaluation is expressed with the \texttt{if expr then expr1 else expr2} expression: if \texttt{expr} evaluates to \texttt{True}, then the result of the expression is \texttt{expr1}, otherwise it is \texttt{expr2}.

These syntax choices were chosen to both simplify expression of the algorithm and ease implementation of the algorithm in a wide variety of programming languages.

\begin{code}

ALGORITHM condense
INPUT: al (grouped association list)
OUTPUT: fdbr (FDBR of al as an association list)

condense alist = map mkpair (groupBy cmp alist)
mkpair list = (head list, map snd list)

fst (a, b) = a
snd (a, b) = b

head (cons x xs) = x
tail (cons x xs) = xs

cmp x y = (fst x) == (fst y)

map f [] = []
map f (cons x xs) = cons (f x) (map f xs)

span p [] =  ([], [])
span p (cons x xs')
 = if (p x)
   then let (ys,zs) = span p xs' in (cons x ys,zs)
   else ([],(cons x xs'))

groupBy eq []          = []
groupBy eq (cons x xs) = cons (cons x ys) (groupBy eq zs)
where (ys,zs) = span (eq x) xs

fdbr = condense al

\end{code}

\begin{theorem}
	%TODO: wording for O(n)
	Algorithm \texttt{condense} has $O(n)$ worst-case time complexity, with comparison on the first element in the association list pairs being the key operation.
\end{theorem}

Proof:

\texttt{cmp} is the function that compares two association list pairs by their first element.  

Let \texttt{al} be a grouped association list.
%Let $n$ denote the number of elements in \texttt{al}.

\texttt{condense al => map mkpair (groupBy cmp al)}.

\begin{proposition}
	The function \texttt{groupBy cmp} has O$(n)$ worst-case time complexity, with comparisons on the elements of the input list using \texttt{cmp} being the key operation.
\end{proposition}

Proof:

Let \texttt{lst = (cons x xs)} be a list with $n$ elements.

In \texttt{groupBy cmp (cons x xs)}, argument \texttt{eq = cmp} and the predicate used in \texttt{span} is \texttt{p = eq x = cmp x}.

The function \texttt{span} returns a partition \texttt{(ys, zs)} of the input list \texttt{xs}, where \texttt{ys} is the longest prefix of \texttt{xs} such that predicate \texttt{p} is \texttt{True} on all elements in the prefix, and \texttt{zs} are the remaining elements in \texttt{xs}.  It follows that \texttt{p} is evaluated at least $s$ times and at most $s + 1$ times, where $s$ is the length of \texttt{pre} (it will only be evaluated $s$ times if \texttt{pre = xs}).

By recursing into the second list returned by \texttt{span}, no previous elements of \texttt{lst} are revisited, and no elements are skipped, so \texttt{groupBy} partitions the input list \texttt{lst} into $m$ lists.  Call this partition \texttt{part}.  Note that the sum of the lengths of all lists in \texttt{part} is $n$.

For each list \texttt{i} in \texttt{part} except the last, \texttt{((length i) - 1) + 1 = length i} comparisons will have been made (\texttt{groupBy} calls span on \texttt{xs}, not \texttt{cons x xs}).  For the last list $last$ in \texttt{part}, \texttt{(length last - 1)} comparisons will have been made (the longest prefix is \texttt{xs} in this case).  Therefore, the total number of comparisons made can be expressed by:

\begin{code}
length part[0] + length part[1] + ... + length part[m - 2]
  + (length part[m - 1] - 1)
= (length part[0] + length part[1] + ... + length part[m - 1]) - 1
= n - 1
\end{code}

Hence, the worst-case time complexity of \texttt{groupBy cmp} is O$(n - 1)$ = O$(n)$, with comparisons using \texttt{cmp} being the key operation.
\qedsymbol

--

\texttt{mkpair} performs no comparisons and therefore has a worst-case time complexity of
$\theta(1)$ with comparisons using \texttt{cmp} being the key operation.

--

The \texttt{map} function evaluates \texttt{mkpair} over every list in the partition returned by (groupBy cmp al).  Since
neither \texttt{map} nor \texttt{mkpair} perform any comparisons using \texttt{cmp}, they do not contribute to the number
of key operations performed.

Therefore, if the length of \texttt{al} is $n$, the worst-case time complexity of \texttt{condense} is \[O(1) + n*(O(1) + O(1)) + O(n) = O(n)\] with comparison on the first element in the association list pairs (\texttt{cmp}) being the key operation.
$\blacksquare$

%Since \texttt{condense} has a worst-case tight bound asymptotic time complexity of $\theta(n)$, it is optimal. \qedsymbol 


\section{Summary}

In this chapter we presented an overview of our query program structure as well as how we efficiently implemented our semantics in Haskell.
We showed the modifications we made to the parser described in \cite{frosthafiz2008} in order to accommodate monadic functions as attributes.
We presented a basic form of query fusion as used by our implementation, and showed an improved version of \texttt{collect} for
grouped association lists.  In Chapter \ref{chapter:timing}, we perform some benchmarks to measure the empirical performance of our code.



\end{document}
