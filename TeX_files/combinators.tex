\documentclass[../main.tex]{subfiles}

\begin{document}

\chapter{Parser Combinators}
\label{chapter:combinators}

To implement our Natural Language Interface, we integrated our semantics with a parser constructed as an executable attribute grammar.  We chose to use the parser
described by Frost and Hafiz in 2008\cite{frosthafiz2008} for this purpose, as it supports top-down parsing of ambiguous grammars.  Our motivations
for choosing this parser were threefold.  First, it enables users to not have to worry about ambiguity or left-recursion in their grammars.  The parser itself
tracks ambiguity and evaluates all unique possible parses\cite{frosthafiz2008}.  Second, both semantic and syntactic rules can be defined together, improving modularity\cite{frosthafiz2008}.
Third, new syntactic and semantic rules can be easily and naturally coded in an attribute grammar that supports left recursion, thereby improving
extensibility\cite{frosthafiz2008}.

We couldn't use the parser
as-is, however, since it didn't support non-referentially transparent functions as attributes.  We detail the modifications we made to the original parser in order to lift this restriction in Section \ref{section:nonrefparser} and in Section \ref{section:nonrefparserimpl}.

\section{Handling non-referentially transparent functions}
\label{section:nonrefparser}

The parser combinators as described in this Thesis differ from their original implementations
as described by Frost and Hafiz\cite{frosthafiz2008}.  The most significant change that we made is that the combinators are now {\em monadic} in nature rather than being strictly pure functions.
Briefly, monads in the Haskell programming language are types that are instances of the {\em Monad} typeclass that obey the
{\em monad laws}\cite{monadlaws}.  In Haskell, functions that are not {\em referentially transparent} are represented using computations in the monad \texttt{IO}, commonly referred to as the {\em IO monad}.

By modifying the parser to work with monadic rather than pure values, the restriction that the semantics themselves must also be pure was lifted, and
we can safely support streaming information from external triplestores in our semantics as a result.
%\texttt{unsafeDupablePerformIO} is no longer needed to ``trick'' the type system into accepting a function is referentially transparent.

The parser works in the IO monad currently, however if other monads were desired, only minimal changes would be required to the parser in order to accommodate
other instances of the Monad typeclass.  In Chapter \ref{chapter:futurework}, one potential application of this functionality is discussed.


\section{Summary of the Parser Combinators}

Aside from the differences noted above, the combinators function the same as they did originally in \cite{frosthafiz2008}.  They are summarized as follows:

\begin{itemize}
	\item \texttt{(<|>)} -- a combinator that represents an alternative.  In the expression ``\texttt{a <|> b}'', first matching \texttt{a} would be tried, and if that parse failed, then matching \texttt{b} would be tried.
	\item \texttt{(*>)} -- a combinator that represents a sequence.  In the expression ``\texttt{a *> b}'', the parser would try to match \texttt{a} followed by \texttt{b}.  Both must be matched in order for the parse to succeed.
\end{itemize}

In Chapter \ref{chapter:implementation}, we show how we constructed a parser as an executable attribute grammar using these combinators and integrated it with our novel event-based denotational semantics to produce a Natural Language Interface to the Semantic Web.

\end{document}