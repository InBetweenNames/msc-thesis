\documentclass[../main.tex]{subfiles}

\begin{document}
	
\chapter{Future Work}

\subsection*{Thoughts on scaling up to handle massive triplestores}

The system presented in this thesis requires event-based triplestores in order to function.  Much of the Semantic Web, however, is not comprised of event-based triplestores.  In order to perform queries on these triplestores, there must be a way to transform these existing triplestores into an event-based form.  In the future, it may be possible to automate this task by using RDF Schemas\cite{owl}  to provide the necessary context to facilitate this conversion.

\subsection*{Efficiency}

In addition, there are several drawbacks with the current implementation that prevent it from being used with massive triplestores.  In the worst case, some of the functions would require reading in a significant amount of data from the triplestore in order to return a value.  One example of this is in the membership functions (provide information).

One promising approach in processing large amounts of data is Conceptual Spaces \cite{cs}, which has already seen some use in performing queries in the Semantic Web \cite{usesofcs}.  It may be possible to develop a new event-based semantics that uses Conceptual Spaces (and by extension Conceptual Geometry) to perform queries on larger datasets.

Another approach could be to ``fuse'' the combinators together when possible, in order to generate
one triplestore request instead of two when possible.

\end{document}