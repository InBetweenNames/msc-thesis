\documentclass[../main.tex]{subfiles}

\begin{document}

\chapter {Event-Based Semantics}

One problem with traditional triplestores is that it is difficult to add contextual information to a triple.  Two common examples of contextual information are time and location.  Many approaches that allow this use a method called {\em reification} \cite{?}.

An event-based triplestore is a special case of a triplestore where information is organized into events.  For example, the triple (``sally'', ``met'', ``susan'') in a traditional triplestore could be represented by three triples in a event-based triplestore:

\begin{code}
	(``event1'', ``type'', ``meet'')
	(``event1'', ``subject'', ``sally'')
	(``event1'', ``object'', ``susan'')
\end{code}

The advantage of this approach is that it is possible to add additional information by simply adding more triples with ``event1'':

\begin{code}
	(``event1'', ``year'', ``1955'')
	(``event1'', ``location'', ``Windsor, Ontario'')
\end{code}

--

Two improvements from the original semantics presented in \cite{frost2014demonstration} \cite{frostagbola2014} are also detailed below.

\subsubsection*{The implicit `and' problem}

\subsubsection*{The use of `by' as a preposition}

\subsubsection*{Efficiency improvements}

The ``collect'' operation now has a worst-case time complexity of O$(n$ log $n)$ time.

\end{document}