\documentclass[../main.tex]{subfiles}

\begin{document}

\chapter {Event-Based Semantics}

\section{Event-Based Triplestores}

One problem with entity-based triplestores is that it is difficult to add contextual information to a triple.  Two common examples of contextual information are time and location.  Many approaches that allow this use a method called {\em reification} \cite{?}.

One form of reification is to organize information into {\em events}.

\begin{definition}[Event]
	A point in time and the physical occurrences that are associated with it \cite{?} -- TODO: find better definition
\end{definition}

For example, the triple \texttt{(sally, met, susan)} in an entity-based triplestore could be represented by three triples:

\begin{code}
	(event1, type, meet)
	(event1, subject, sally)
	(event1, object, susan)
\end{code}

These triples describe the event in which ``Sally met Susan'' rather than directly describing the meeting itself.  The advantage of this approach is that it is possible to add additional information about the meeting by simply adding more triples with \texttt{event1} as the {\em subject}:

\begin{code}
	(event1, year, 1955)
	(event1, location, windsor)
\end{code}

Triplestores that organize their information in this fashion are called {\em Event-based Triplestores}.

\begin{definition}[Event-Based Triplestore]
	A triplestore where the \texttt{subject} of the triples contained within it refer to Events\cite{frostagboola2014}
\end{definition}

The key motivation behind using Event-based triplestores in this Thesis is that they directly support reification on triples\cite{frostagboola2014}.

\section{Original Event-Based Denotational Semantics}

The semantics in this Thesis is based on work that was originally described in \cite{frost2013event}.  That work was later improved upon in \cite{frostagboola2014}.


\begin{definition}[Image]
\end{definition}

\section{Improvements over Original Semantics}

The improvements from the original semantics presented in \cite{frost2014demonstration} \cite{frostagboola2014} are detailed below.

\subsection{Naming and definition of ``Images''}

Originally, semantic functions were defined using {\em Images} as their arguments and return values.

\subsection{Type safety}

\subsection{The implicit `and' problem}

\subsection{The use of `by' as a preposition}

\subsection{Efficiency}

The ``collect'' operation now has a worst-case time complexity of O$(n$ log $n)$ time.

\end{document}