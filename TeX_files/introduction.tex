\documentclass[../main.tex]{subfiles}

\begin{document}

\chapter{Introduction}

%INTRODUCTION:

\section{Motivation}

Gone are the days where we could assume that a toaster was just a toaster, a refrigerator was just a refrigerator, or a kettle was just a kettle.
Increasingly, Internet-aware ``smart'' devices are becoming the norm, silently replacing their ordinary counterparts with versions that contain microprocessors,
sensors, network adapters, and other electronic hardware. These devices are what comprise the {\em Internet of Things}, a revolutionary change in how we interact with devices and appliances around us.  One benefit of this change is that these devices are able to be operated remotely and through alternative interfaces, drastically
improving accessibility for those who have disabilities.  Natural Language Interfaces, in particular, could carry tremendous value to these users.

The Internet of Things, or {\em IoT}, is built on {\em The Semantic Web}, a body of open standards that allows these devices to ``speak the same language'' to one another, much
like how the {\em World Wide Web} operates today\cite{semwebanalogy}.  If it were possible to design a framework for building Natural Language Interfaces to the Semantic Web,
Natural Language Interfaces could in turn be designed for these IoT-enabled devices.


%TODO: Add more here?  Tie in Semantic Web with Natural Language, and talk about how NLP interfaces could be used to control IoT devices

\section{The Semantic Web}

To begin understanding The Semantic Web, first some terminology must be introduced.

\begin{definition}[Uniform Resource Identifier]
	``A compact sequence of characters that identifies an abstract or physical resource''\cite{w3curi}
\end{definition}

A Uniform Resource Identifier may also be referred to as a {\em URI}.  A URI may take the form of a {\em name}, a {\em location}, or possibly both at once.
A common example of a URI is an HTTP URL for a website.  Another example would be the ISBN of a book, or a telephone number.
URIs are a fundamental component of The Semantic Web, as it standardizes the notion of identification for the devices, or more generally speaking, {\em agents}
that comprise it.  Having a canonical name for a given object helps agents to refer to it, and in turn understand what one another are referring to when communicating.

\begin{definition}[Triple]
	A 3-tuple that has the form $(\text{\em subject}, \text{\em predicate}, \text{\em object})$, where {\em subject}, {\em predicate}, and {\em object} are Uniform Resource Identifiers\cite{w3csemanticweb}.
\end{definition}

Fundamentally, The Semantic Web is a network of online databases that store facts in the form of {\em triples}.
These {\em triples} compose the basic elements of the Resource Description Framework data model that underlies the Semantic Web, and databases that contain them are commonly referred to as {\em triplestores} or {\em RDF triplestores}.  When we refer to the \texttt{subject}, \texttt{predicate}, or \texttt{object} of a triple, we are referring to the first, second, or third component of that triple, respectively.

Traditionally, triplestores describe facts in terms of {\em entities}.

\begin{definition}[Entity]
	``A thing capable of an independent existence that can be uniquely identified''\cite{kent2015era}
\end{definition}

We call triplestores whose information is organized around entities {\em entity-based triplestores}.

\begin{definition}[Entity-based Triplestore]
	A triplestore where the {\em subject} and {\em object} of the triples contained within it refer only to {\em entities}
\end{definition}

In this Thesis Report, we use the syntax \texttt{(s, v, o)} to describe a triple.  However, in practice, different encodings are used
to represent triples.  For instance, in {\em N-Triples} syntax\cite{w3cntriples}, the triple \texttt{(s, v, o)} would be represented using
the string obtained by substituting the URIs that \texttt{s}, \texttt{v}, and \texttt{o} represent directly into their corresponding placeholders ``\texttt{\$s}'', ``\texttt{\$v}'', and ``\texttt{\$o}'' in the following string:  ``\texttt{<\$s> <\$v> <\$o>.}''.  In our Haskell code, we would represent this triple using the tuple \texttt{(ss, vs, os)}, where \texttt{ss}, \texttt{vs}, and \texttt{os} are, respectively, the URIs that \texttt{s}, \texttt{v}, and \texttt{o} represent mapped to the \texttt{String} type in the language.

Therefore, in this Thesis Report the triples we refer to are ``abstract'' triples rather than the actual concrete representations of those triples in practice.

%THE PROBLEM:

\section {The Problem}

Although information represented in the Semantic Web is standardized by the W3C\cite{w3csemanticweb}, the problem of how to actually query information contained within it is still an area of active research.

The SPARQL Protocol and RDF Query Language is an attempt to provide an SQL-like interface to the Semantic Web.  Currently, it is the de-facto method of querying RDF triplestores.  SPARQL queries form patterns that define restrictions on the elements of triples.  Only triples matching the query pattern are returned in the result.  Users submit queries to a {\em SPARQL endpoint} which in turn executes the query against a triplestore and returns the results.  SPARQL is not intended to be used directly by humans, however. User-friendly Semantic Web query interfaces provide higher level metaphors for interacting with RDF triplestores, improving accessibility.  These query interfaces then transform these metaphors into corresponding SPARQL queries.

Querying the Semantic Web using Natural Language is an active area of research interest, as it allows users with little to no technical background to construct queries for RDF triplestores.  This allows, for instance, health or police databases to be queried by professionals with minimal effort by the user.  It also enhances accessibility of The Semantic Web for users who have disabilities.  As there have been several attempts to construct a Natural Language Interface for the Semantic Web, and research is ongoing, the problem of doing so is non-trivial.  A summary of existing approaches for Natural Language Interfaces to query the Semantic Web is given below:


%SUMMARY OF EXISTING APPROACHES:
\section {Existing approaches}

\subsubsection{ORAKEL}
An ontology-aware English interface to the Semantic Web based on Montague semantics\cite{cimiano2007orakel}.  ORAKEL parses sentences according to a provided grammar,
and evaluates queries based on a compositional semantics.  It supports quantification, negation, and conjunction in queries.  ORAKEL directly attempts to convert the input query to a SPARQL query.

\subsubsection{QuestIO}
An ontology-aware English interface to the Semantic Web that is keyword oriented, attempting to match words against concepts to hone down queries\cite{tablan2008natural} .  Uses SeRQL\cite{serql}, a query language similar to SPARQL, to form queries.  Sentences are transformed into SeRQL queries by the use of a formal semantics.

\subsubsection{AutoSPARQL}
A supervised machine learning approach using English to query the Semantic Web\cite{lehmann2011autosparql}.  Queries are provided in the form of keywords, which are used to construct query trees.  These are then converted to SPARQL queries.  It is a feedback oriented system in that the user is expected to be actively involved in refining subsequent results by selecting candidates from the returned set that best match what the user is looking for.

%THE PROBLEM WITH EXISTING APPROACHES:

\section {Shortcomings of previous approaches}

%These approaches all attempt to directly translate the user's query into a SPARQL query for processing.
While these approaches have seen success, they all share a shortcoming in that they do not allow for prepositional phrases in queries, and therefore have limited coverage of the English language.

\section {New approach}
%OVERVIEW OF NEW APPROACH:

The work presented in this thesis draws on two main concepts: executable attribute grammars\cite{frosthafiz2008} and event-based denotational semantics\cite{frostagboola2014}.

Executable attribute grammars are a natural way to implement Natural Language processors\cite{?}, and since they allow top-down rather than bottom-up parsing, they are highly modular\cite{frosthafiz2008}.  This makes them well suited to the natural specification of semantic rules.

Event-based denotational semantics operate on event-based triplestores\cite{frost2014demonstration} rather than entity-based triplestores.  Triplestores describe entities and their relations to other entities, but a problem exists in how to add contextual information to a particular triple.  In an event-based triplestore, triples describe events rather than entities, and information about entities and their relationships to one another may be gleaned from the events in which they occur.  Additional information about an event may be added by simply adding a new triple to the triplestore.

As an example, the sentence ``Jane bought a pencil'' could be represented in an entity-based triplestore with the triple


\begin{code}
	(Jane, purchased, pencil_1)
\end{code}

In an event-based triplestore, the same sentence could be represented with three triples:

\begin{code}
	(event1, subject, Jane)
	(event1, type, purchase)
	(event1, object, pencil_1)
\end{code}
	
Additionally, other information about the event may be added as well, for example including the purchase price with the triple: \texttt{(event1, cost, \$1)}, or perhaps the time $t$ the transaction occurred with \texttt{(event1, time, $t$)}.  

\section{Thesis Statement}
By integrating a novel event-based denotational semantics with a parser constructed as an executable attribute grammar, it is possible to create a highly modular and extensible Natural Language Interface to the Semantic Web that supports the use of prepositional phrases in queries.

\section{Proof of Concept} We prove the Thesis by creating an online English query interface to a triplestore containing thousands of facts about the solar system\cite{Solarman:2016}.

Some example queries that can be handled by this system include:

\begin{itemize}
	\item ``when was something discovered at mt\_wilson''
	\item ``how was the thing that was discovered at flagstaff discovered''
	\item ``what was discovered in 1877 in us\_naval\_observatory''
	\item ``what planet is orbited by a moon that was discovered in 1684''
	\item ``which vacuumous moon that orbits jupiter was discovered by nicholson or hall with a telescope in 1938 in mt\_wilson or mt\_hopkins''
\end{itemize}

In addition to this, we demonstrate a novel method of handling the word ``by'' as used in the passive form of a verb by treating it directly as a preposition within our grammar, unifying our treatment of active and passive verbs.  In doing this, we are able to accommodate queries such as ``which moon was discovered in 1877 by hall'' without any added complexity to the semantics.

\section{Structure of Thesis Report}

The remainder of this Thesis Report is structured as follows:

\begin{enumerate}
	\item Demonstration
	\item The event-based semantics
	\item The parser combinator
	\item The query program
	\item Timing
	\item Thoughts on scaling up to handle massive triplestores
	\item Proof of the Thesis
	\item Conclusions
\end{enumerate}

\end{document}