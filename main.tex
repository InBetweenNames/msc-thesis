\documentclass[oneside, 12pt]{book}

\usepackage[top=1in, bottom=1in, left=1.5in, right=1in]{geometry}
\usepackage{subfiles}
\usepackage[nottoc,numbib]{tocbibind}
\usepackage{listings}
\usepackage{setspace}
\usepackage{amsthm}
\usepackage{amssymb,amsmath}
\usepackage{url}
\usepackage{verbatim}
\usepackage{graphicx}
\usepackage{tabularx}
\usepackage{mathptmx}% http://ctan.org/pkg/mathptmx
\usepackage{fancyhdr}
\usepackage{setspace}
\usepackage[mark]{gitinfo2}
\usepackage[hidelinks]{hyperref}
\usepackage[backend=biber,bibstyle=numeric-comp,sorting=ydnt,natbib=true,mcite=true,maxnames=100,url=true,isbn=false,doi=false,uniquename=init,giveninits=true,hyperref=true,backref=true,date=edtf,sortcites]{biblatex}

\addbibresource{thesis.bib}

%TODO: Change "plain" to "revision"
\fancypagestyle{revision}{
	\fancyhf{}
	\fancyhead[L]{\nouppercase{\leftmark}}
	\fancyhead[R]{\thepage}
	%\fancyfoot[L]{Revision 3}
}

%TODO: change "plain" to "revision"
\pagestyle{revision}

\newtheorem{theorem}{Theorem}[section]
\newtheorem{lemma}[theorem]{Lemma}
\newtheorem{proposition}[theorem]{Proposition}
\newtheorem{corollary}[theorem]{Corollary}

\lstloadlanguages{Haskell}

\lstnewenvironment{code}
{\lstset{}%
	\csname lst@SetFirstLabel\endcsname}
{\csname lst@SaveFirstLabel\endcsname}
\lstset{
	basicstyle={\scriptsize\ttfamily\singlespacing},
	flexiblecolumns=false,
	basewidth={0.5em,0.45em},
	literate={+}{{$+$}}1 {/}{{$/$}}1 {*}{{$*$}}1 {=}{{$=$}}1
	{>}{{$>$}}1 {<}{{$<$}}1 {\\}{{$\lambda$}}1
	{\\\\}{{\char`\\\char`\\}}1
	{->}{{$\rightarrow$}}2 {>=}{{$\geq$}}2 {<-}{{$\leftarrow$}}2
	{<=}{{$\leq$}}2 {=>}{{$\Rightarrow$}}2
	{\ .\ }{{$\circ$}}2
	{>>}{{>>}}2 {>>=}{{>>=}}2
	{|}{{$\mid$}}1
}

\lstnewenvironment{spec}
{\lstset{}%
	\csname lst@SetFirstLabel\endcsname}
{\csname lst@SaveFirstLabel\endcsname}
\lstset{
	basicstyle={\scriptsize\ttfamily\singlespacing},
	flexiblecolumns=false,
	basewidth={0.5em,0.45em},
	literate={+}{{$+$}}1 {/}{{$/$}}1 {*}{{$*$}}1 {=}{{$=$}}1
	{>}{{$>$}}1 {<}{{$<$}}1 {\\}{{$\lambda$}}1
	{\\\\}{{\char`\\\char`\\}}1
	{->}{{$\rightarrow$}}2 {>=}{{$\geq$}}2 {<-}{{$\leftarrow$}}2
	{<=}{{$\leq$}}2 {=>}{{$\Rightarrow$}}2
	{\ .\ }{{$\circ$}}2
	{>>}{{>>}}2 {>>=}{{>>=}}2
	{|}{{$\mid$}}1
}

\newcommand{\uwinverytightsinglespacelen}{0.9}
\newcommand{\uwintightsinglespacelen}{1.0}
\newcommand{\uwinsinglespacelen}{1.1}
\newcommand{\uwinonehalfspacelen}{1.5}
\newcommand{\uwindoublespacelen}{2.0}
\newcommand{\uwinlistofspacelen}{1.5}
\newcommand{\uwindefaultspacelen}{\uwindoublespacelen}

\newcommand{\uwinverytightsinglespace}%
{\linespread{\uwinverytightsinglespacelen}}
\newcommand{\uwintightsinglespace}%
{\linespread{\uwintightsinglespacelen}}
\newcommand{\uwinsinglespace}%
{\linespread{\uwinsinglespacelen}}
\newcommand{\uwinonehalfspace}%
{\linespread{\uwinonehalfspacelen}}
\newcommand{\uwindoublespace}%
{\linespread{\uwindoublespacelen}}
\newcommand{\uwinlistofspace}%
{\linespread{\uwinlistofspacelen}}
\newcommand{\uwindefaultspace}%
{\linespread{\uwindefaultspacelen}}

\newenvironment{uwinverytightsinglespaceenv}%
{\begin{spacing}{\uwinverytightsinglespacelen}}%
	{\end{spacing}}
\newenvironment{uwintightsinglespaceenv}%
{\begin{spacing}{\uwintightsinglespacelen}}%
	{\end{spacing}}
\newenvironment{uwinsinglespaceenv}%
{\begin{spacing}{\uwinsinglespacelen}}%
	{\end{spacing}}
\newenvironment{uwinonehalfspaceenv}%
{\begin{spacing}{\uwinonehalfspacelen}}%
	{\end{spacing}}
\newenvironment{uwindoublespaceenv}%
{\begin{spacing}{\uwindoublespacelen}}%
	{\end{spacing}}
\newenvironment{uwinlistofspaceenv}%
{\begin{spacing}{\uwinlistofspacelen}}%
	{\end{spacing}}
\newenvironment{uwindefaultspaceenv}%
{\begin{spacing}{\uwindefaultspacelen}}%
	{\end{spacing}}

\long\def\ignore#1{}

\author{Shane Peelar}
\title{Accommodating prepositional phrases in a highly modular
	natural language query interface to semantic web triplestores using a novel event-based denotational semantics for English and a set of parser combinators.}
\date{December 2016}

\begin{document}
	
\renewcommand{\gitMark}{
	Branch: \gitBranch\,@\,\gitAbbrevHash{} 
	\textbullet{} 
	Release:\gitReln{} 
	(\gitAuthorDate)
}
	
\pagenumbering{roman}

%---------

\clearpage

\thispagestyle{empty}
\begin{center}
	\vspace*{1in}
	
	\begin{uwinonehalfspaceenv}
		\Large\textbf{Accommodating prepositional phrases in a highly modular
			natural language query interface to semantic web triplestores using a novel event-based denotational semantics for English and a set of functional parser combinators}
	\end{uwinonehalfspaceenv}
	
	\vspace{\fill} %
	\begin{uwinonehalfspaceenv}
		By:\\*
		Shane Peelar
	\end{uwinonehalfspaceenv}
	\vspace{\fill}
	
	\normalsize
	A Thesis \\*
	Submitted to the Faculty of Graduate Studies \\*
	through the School of Computer Science \\*
	in Partial Fulfillment of the Requirements for \\*
	the degree of Master of Science at the \\*
	University of Windsor \\*
	
	\vspace{1in}
	Windsor, Ontario, Canada \\
	\vspace{0.5cm}
	2016 \\
	\vspace{0.5cm}
	\textcopyright \  2016 Shane Peelar
\end{center}

%---------


\clearpage
\thispagestyle{empty}

\vspace*{\fill}

%\currentpdfbookmark{Copyright}{copyrightpage}%
\noindent \textcopyright{} 2016, Shane Peelar

\vspace{2ex}

\noindent All Rights Reserved. Absolutely no part of this document may
be reproduced, stored in a retrieval system, translated, in any form
or by any means electronic, mechanical, facsimile, photocopying, or
otherwise, without the prior written permission of the copyright
holder.

\vspace*{\fill}

%---------


%----------

\clearpage
\chapter*{Abstract\markboth{\MakeUppercase{Abstract}}{}}
\addcontentsline{toc}{chapter}{Abstract}


%----------

\clearpage
\chapter*{\centering Dedication\markboth{\MakeUppercase{Dedication}}{}}
\addcontentsline{toc}{chapter}{Dedication}

%----------

\clearpage
\chapter*{Acknowledgements\markboth{\MakeUppercase{Acknowledgements}}{}}
\addcontentsline{toc}{chapter}{Acknowledgements}

%----------

\tableofcontents

%---------

\clearpage
% \phantomsection
\addcontentsline{toc}{chapter}{Declaration of Co-Authorship / Previous Publication}
\chapter*{Declaration of Co-Authorship / Previous Publication\markboth{\MakeUppercase{Declaration of Co-Authorship / Previous Publication}}{}}

\begin{uwindefaultspaceenv}
	I hereby declare that this thesis incorporates material that is result of joint research, as follows:
	
	Some of the material in this thesis is derived from the following research papers:
\end{uwindefaultspaceenv}

\cite{frosthafiz2008} \fullcite{frosthafiz2008}

\cite{frostagboola2014} \fullcite{frostagboola2014}

\cite{frost2014demonstration} \fullcite{frost2014demonstration}

\begin{uwindefaultspaceenv}
	%TODO
	Paper \cite{frosthafiz2008} (Frost, Hafiz, and Callaghan) describes a set of functional parser combinators developed by Frost and Hafiz as part of Hafiz's doctoral thesis work, which enables language processors to be built as executable specifications of fully-general attribute grammars, including ambiguous left-recursive grammars.  The processors use a polynomial time complexity top-down parsing strategy which enables a natural specification of the grammars and the associated semantic rules.  This was previously thought to be impossible, and was stated as such in many textbooks on parsing.
	
	Paper \cite{frostagboola2014} (Frost, Agboola, Matthews, and Donais) describes an event based semantics developed by Dr. Frost and his research team and includes extracts from a Haskell program which demonstrated the viability of the semantics with respect to an in-program database of triples coded as part of the program.
	
	Paper \cite{frost2014demonstration} (Frost, Donais, Matthews, Agboola, and Stewart) describes the demonstration of the Haskell program which I wrote and which forms the basis of this thesis work.  The reason that I am not listed as an author is that hte paper was submitted before I officially joined the research team.  I developed the Haskell program after the paper was submitted.  The online program was the one used by Dr. Frost in the demonstration he gave at the conference this paper was presented at.
	
	My contributions to the research project include:
	
	\begin{itemize}
		\item Improving the efficiency of the programs which implement the event-based semantics
		\item Integrating the event-based semantics with the parser combinators to build the query processor
		\item Enhancing the existing module to access the external triplestore with efficient methods to do so, including a basic form of query fusion in the form of memoization
		\item Demonstrating a novel method of handling the word ``by'' in prepositional phrases
		\item Building a web interface to the query processor which includes both an English Natural Language Interface and also a safe Direct Query Interface for directly evaluating the combinators
		\item Converting the parser Hafiz wrote\cite{frosthafiz2008} to work natively work with monads in Haskell, as well as the original semantics \cite{frost2014demonstration} to be monad based
		\item Maintaining the XSaiga package on {\em Hackage} \cite{XSaiga:2016}, an online repository of Haskell libraries and programs, which contains the semantics, parser, and triplestore described in this Thesis
	\end{itemize}
	
	I am aware of the University of Windsor Senate Policy on Authorship and I certify that I have properly acknowledged the contribution of other researchers to my thesis, and have obtained written permission from each of the co-author(s) to include the above material(s) in my thesis. 
	
	I certify that, with the above qualification, this thesis, and the research to which it refers, is the product of my own work.
	
\end{uwindefaultspaceenv}

\listoffigures

\clearpage

\listoftables
\clearpage

\pagenumbering{arabic}

\subfile{./TeX_files/introduction}
\subfile{./TeX_files/demonstration}
\subfile{./TeX_files/semantics}
\subfile{./TeX_files/combinators}
\subfile{./TeX_files/interface}
\subfile{./TeX_files/timing}
\subfile{./TeX_files/future}
\subfile{./TeX_files/conclusions}

% bibliography, glossary and index would go here.

\printbibliography[heading=bibintoc]

\chapter*{Vita Auctoris}
\addcontentsline{toc}{chapter}{Vita Auctoris}

\end{document}