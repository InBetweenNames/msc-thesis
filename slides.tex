% Load the beamer-sharcnet document class...
\documentclass[tabu]{beamer-uwindsor}

% Have latex inform when another run is needed...
\usepackage{rerunfilecheck}

% Use the csquotes package...
\usepackage{csquotes}

% Use the listings package...
\usepackage{listings}

\lstloadlanguages{Haskell}

\lstnewenvironment{code}
{\lstset{}%
	\csname lst@SetFirstLabel\endcsname}
{\csname lst@SaveFirstLabel\endcsname}
\lstset{
	basicstyle={\scriptsize\ttfamily\singlespacing},
	flexiblecolumns=false,
	basewidth={0.5em,0.45em},
	literate={+}{{$+$}}1 {/}{{$/$}}1 {*}{{$*$}}1 {=}{{$=$}}1
	{>}{{$>$}}1 {<}{{$<$}}1 {\\}{{$\lambda$}}1
	{\\\\}{{\char`\\\char`\\}}1
	{->}{{$\rightarrow$}}2 {>=}{{$\geq$}}2 {<-}{{$\leftarrow$}}2
	{<=}{{$\leq$}}2 {=>}{{$\Rightarrow$}}2
	{\ .\ }{{$\circ$}}2
	{>>}{{>>}}2 {>>=}{{>>=}}2
	{|}{{$\mid$}}1
}

\lstnewenvironment{spec}
{\lstset{}%
	\csname lst@SetFirstLabel\endcsname}
{\csname lst@SaveFirstLabel\endcsname}
\lstset{
	basicstyle={\scriptsize\ttfamily\singlespacing},
	flexiblecolumns=false,
	basewidth={0.5em,0.45em},
	literate={+}{{$+$}}1 {/}{{$/$}}1 {*}{{$*$}}1 {=}{{$=$}}1
	{>}{{$>$}}1 {<}{{$<$}}1 {\\}{{$\lambda$}}1
	{\\\\}{{\char`\\\char`\\}}1
	{->}{{$\rightarrow$}}2 {>=}{{$\geq$}}2 {<-}{{$\leftarrow$}}2
	{<=}{{$\leq$}}2 {=>}{{$\Rightarrow$}}2
	{\ .\ }{{$\circ$}}2
	{>>}{{>>}}2 {>>=}{{>>=}}2
	{|}{{$\mid$}}1
}

%
% Use TrueType fonts (i.e., xelatex).
%
% Use "Charis SIL" font even though it is not sans-serif as
% it is easy-to-read, has all Unicode characters, is under
% the open SIL font licence, and is free.
%
% As SIL does not have a monospaced font, Liberation Mono
% is used instead.
%
\usepackage{xltxtra,fontspec,xunicode}
\defaultfontfeatures{Scale=MatchLowercase}
\setromanfont{Charis SIL}
\setsansfont{Charis SIL}
\setmainfont{Charis SIL}
%\setmonofont[Scale=0.70]{Liberation Mono}
\setmonofont{Liberation Mono}

%
% Define slide show and presenter-specific globals...
%
\providecommand{\basetexfilename}{slides}
\providecommand{\emailshane}{peelar@uwindsor.ca}
\title[Masters Thesis Proposal]{Accommodating prepositional phrases in a highly modular
	natural language query interface to semantic web triplestores using a novel event-based denotational semantics for English and a set of parser combinators.}
\date{September 12th 2016}
\author[Shane Peelar]{%
  Shane Peelar, B.Sc. \texorpdfstring{\\}{}%
  {\footnotesize\href{mailto:\emailshane}{\emailshane}} \texorpdfstring{\\}{}%
}
\institute[University of Windsor]{%
  University of Windsor, \texorpdfstring{\\}{}%
  Windsor, Ontario, Canada \texorpdfstring{\\}{}%
  Copyright \copyright{} 2016 \insertshortauthor{}. All Rights Reserved. \texorpdfstring{\\}{}%
  ~ \texorpdfstring{\\}{}%
}
\subject{Intro}

\usepackage{hyperref}

\usepackage{xcolor}
\hypersetup{
	colorlinks,
	linkcolor={red!50!black},
	citecolor={blue!50!black},
	urlcolor={blue!80!black}
}

% bibliography...
\usepackage[backend=biber,bibstyle=numeric-comp,sorting=ydnt,natbib=true,mcite=true,maxnames=100,url=true,isbn=false,doi=false,uniquename=init,giveninits=true,hyperref=true,backref=true,date=edtf,sortcites]{biblatex}
\renewcommand{\subtitlepunct}{\addcolon\addspace}

%\hypersetup{pdfpagemode=FullScreen}


% Load support for hypertext references...
\usepackage{url}

% Turn off the navigation bar...
\beamertemplatenavigationsymbolsempty

% Support arithmetic with numbers...
\usepackage{fp}

% Use Tikz
\usepackage{tikz}

%
% Make it easy to output a number in roman numeral form...
%
\makeatletter
\newcommand*{\asroman}[1]{\expandafter\@slowromancap\romannumeral #1@}
\makeatother

%
% Redefine the part page to subtract one from the part # since the first
% part of this document represents the document title page and overview
% preamble...
%
\setbeamertemplate{part page}
{
  \begin{centering}
    \FPeval{\result}{clip(\insertpartnumber-1)}
    {\usebeamerfont{part name}\usebeamercolor[fg]{part name}\partname~\asroman{\result}}
    \vskip1em\par
    \begin{beamercolorbox}[sep=16pt,center,shadow=false,rounded=true]{part title}
      \usebeamerfont{part title}\insertpart\par
    \end{beamercolorbox}
  \end{centering}
}

%
% Define how beamer shows auto-generated pages...
%
\makeatletter
\newcommand*{\resetsectioncount}{
  \beamer@tocsectionnumber=0\relax%
  \setcounter{section}{0}%
}
\makeatother

\makeatletter
\AtBeginPart{%
  % Restart section numbering within each part...
  \resetsectioncount%
  % And output (if applicable) the part page...
  \ifthenelse{\boolean{showpartpage}}{%
    \frame{\partpage}%
  }{}%
}
\makeatother

\AtBeginSection[]{%
  \ifthenelse{\boolean{showsectiontoc}}{%
    \begin{frame}{Table of Contents}%
      \tableofcontents[sectionstyle=show/hide,subsectionstyle=show/show/hide]%
    \end{frame}%
  }{}%
}

% Bibliography...
\addbibresource{thesis.bib}

% Title Page Logo
\titlegraphic{\includegraphics[width=2cm]{logos/uwin-logo-horz-3col.eps}}

% 
% The presentation slide content follows...
%
\begin{document}
	% Don't show logo or footer on title page...
	{
	\setbeamertemplate{logo}{}
	\setbeamertemplate{footline}{} 
	\begin{frame}[label=TitleSlide,noframenumbering]
	  \titlepage
	\end{frame}
	}

	\sectionwithouttoc{Outline}
	\begin{frame}{Outline}
		\tableofcontents
	\end{frame}
	
	\sectionwithouttoc{Introduction}
	\begin{frame}{Introduction}
		\begin{itemize}
			\item The Semantic Web (part of Web 3.0)
			\begin{itemize}
				\item Distributed and decentralized
				\item Resource Description Framework (RDF) \cite{w3csemanticweb}
				\item SPARQL Protocol and RDF Query Language (SPARQL)
			\end{itemize}
			\item Can be thought of, informally, as applying the model of the World Wide Web to databases
			\item SPARQL is low-level (like SQL)
		\end{itemize}
	\end{frame}
	
	\subsectionwithouttoc{Resource Description Framework}
	\begin{frame}{Resource Description Framework}
		\begin{itemize}
			\item Data is organized into {\em triples}
			\item A {\em triple} is a 3-tuple that has the form $(\text{\em subject}, \text{\em predicate}, \text{\em object})$
			\item {\em subject}, {\em predicate}, and {\em object} are Uniform Resource Identifiers (URIs) \cite{w3csemanticweb}
			\item A database containing these triples is commonly called a ``triplestore''
			\item Example triple: (``Jane'', ``purchased'', ``pencil'')
		\end{itemize}
	\end{frame}
	
	\subsectionwithouttoc{SPARQL}
	\defverbatim[colored]\lstI{
		\begin{lstlisting}[basicstyle=\ttfamily,keywordstyle=\color{red}]
	PREFIX foaf:  <http://xmlns.com/foaf/0.1/>
	SELECT ?name ?email
	FROM <http://www.w3.org/People/Berners-Lee/card>
	WHERE {
	{      SELECT DISTINCT ?person ?name WHERE { 
	?person foaf:name ?name 
	} ORDER BY ?name LIMIT 10 OFFSET 10    }
	}
		\end{lstlisting}
	}
	
	\begin{frame}{SPARQL Protocol and RDF Query Language}
		\begin{itemize}
			\item RDF Query Language
			\item Similar to SQL
			\item Queries are submitted to {\em SPARQL endpoints}
			\item Form patterns that define restrictions on the elements of triples
			\item Similar to pattern matching in functional programming
			\item Low-level
			\item Example: \lstI
		\end{itemize}
	\end{frame}
	
	\sectionwithouttoc{Natural Language Interfaces to the Semantic Web}
	\begin{frame}{Natural Language Interfaces to the Semantic Web}
		\begin{itemize}
			\item SPARQL is low level and not user-friendly
			\item Natural Language Interfaces
			\begin{itemize}
				\item Require little technical knowledge to use
				\item Require minimal effort on part of the user
				\item Area of active research, with several previous attempts
			\end{itemize}
			\item ORAKEL \cite{cimiano2007orakel} -
			\item QuestIO \cite{tablan2008natural} -
			\item AutoSPARQL \cite{lehmann2011autosparql} -
		\end{itemize}
	\end{frame}
	
	\sectionwithouttoc{Problem Statement}
	\begin{frame}{Problem Statement}
		
		\textbf{The Problem}:
		Previous work has seen success, but the English NLIs have a shortcoming in that they do not allow for prepositional phrases in queries, and hence have a limited coverage of the English language.
		\newline
		
		The work presented in this thesis draws on two main concepts:
		\begin{itemize}
			\item Executable Attribute Grammars\cite{frosthafiz2008}
			\item Event-Based Denotational Semantics\cite{frostagbola2014}.
		\end{itemize}
		
	\end{frame}
	
	\sectionwithouttoc{Executable Attribute Grammars}
	\begin{frame}{Executable Attribute Grammars}
		\begin{itemize}
		\item Executable attribute grammars are a natural way to process Natural Language queries, and since they allow top-down rather than bottom-up parsing, they are highly modular \cite{frosthafiz2008}.
		\end{itemize}
	\end{frame}
	
	\sectionwithouttoc{Event-Based Denotational Semantics}
	\begin{frame}{Event-Based Denotational Semantics}
		\begin{itemize}
		\item Event-based denotational semantics operate on event-based triplestores \cite{frost2014demonstration} rather than traditional triplestores
		\end{itemize}
	\end{frame}
	
	\sectionwithouttoc{Thesis Statement}
	\begin{frame}{Thesis Statement}
		By integrating a novel event-based denotational semantics with a parser constructed as an executable attribute grammar, it is possible to create a highly modular and extensible Natural Language Interface to the Semantic Web that supports the use prepositional phrases in queries.
	\end{frame}
	
	\sectionwithouttoc{Proof of Concept}
	\begin{frame}{Proof of Concept}
		We prove the Thesis by creating an online English query interface to a triplestore containing thousands of facts about the solar system. \cite{Solarman:2016}.
	\end{frame}
	
	\sectionwithouttoc{Nove}
	\begin{frame}{Proof of Concept}
	\end{frame}
	
	\sectionwithouttoc{Timeline}
	\begin{frame}{Timeline}
		
		\begin{itemize}
			\item September - Revamp web interface
			\item October - Complexity analysis and report writing
			\item November - Investigate how RDF schemas could be used to convert existing triplestores to event-based
			\item December - Completion and defence 
		\end{itemize}
		
	\end{frame}
		
	
	\sectionwithouttoc{Thank You \& Feedback}
	\begin{frame}{Thank You \& Feedback}
		\centering
		Thank you for attending! \\ Comments, suggestions, or questions?
	\end{frame}
  
	\sectionwithouttoc{References}
	\begin{frame}[allowframebreaks]{References}
	  \def\newblock{}
	  \setbeamertemplate{bibliography item}[text]
	  \setlength\bibitemsep{1.5\itemsep}
	  \printbibliography
	\end{frame}
	
	%\appendix 
	%\section*{Appendix Overview}
	%\begin{frame}{Appendix Overview}
	%  \tableofcontents[part=3,sectionstyle=show/show,subsectionstyle=hide/hide/hide]
	%\end{frame}
	
	%\sectionwithouttoc{An Appendix Title}
	%\begin{frame}{\insertsection}
	%  This is the first section.
	%\end{frame}
	
	%\sectionwithouttoc{Another Appendix Title}
	%\begin{frame}{\insertsection}
	%  This is the second section.
	%\end{frame}
\end{document}
